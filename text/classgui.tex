\documentclass[11pt,a4paper,dvipdfmx,titlepage]{jsreport}
\usepackage{tcolorbox}
\tcbuselibrary{breakable}
\usepackage{moreverb}
\usepackage{graphicx}
\usepackage{longtable}
\usepackage{varwidth}
\usepackage{amsmath}
\newcommand{\kara}{}

\usepackage{makeidx}
\makeindex

\newcounter{placounter} 

\setcounter{secnumdepth}{3}
\setcounter{tocdepth}{3}   
%凡例用BOX
\newtcolorbox[auto counter, number within=chapter, 
number freestyle={\noexpand\thechapter.\noexpand\arabic{\tcbcounter}}]{legbox}[2][]{% 
colframe=black!80,colback=white,colbacktitle=black!30,fonttitle=\bfseries\sffamily,breakable,
    title=コード ~\thetcbcounter: #2,     #1 } 
%コード入力用BOX
\newtcolorbox[]{codebox}[2][]{%
colframe=black!80,colback=white,colbacktitle=black!30,fonttitle=\bfseries\sffamily,breakable,
title= #2 }
%入れ子(Nested)改ページありのコード入力用
\newtcolorbox[]{codebox2}[2][]{%
colframe=black!80,colback=white,colbacktitle=black!30,fonttitle=\bfseries\sffamily,
shrink break goal=0pt,
enforce breakable=true,title= #2 }
%練習問題用BOX
\newtcolorbox[use counter=placounter]{plabox}[2][]{%
colframe=black!80,colback=white,colbacktitle=black!90,sharp corners,fonttitle=\bfseries\sffamily,breakable,
colback=black!5!white,
title=練習問題 \if #1 \else : #1 \fi #2 \thetcbcounter,#1}
%文法記載用BOX
\newtcolorbox[auto counter,number within=section]{grabox}[2][]{%
colframe=black!20,colback=white,colbacktitle=black!60,fonttitle=\bfseries\sffamily,breakable,
title=~\thetcbcounter \if #1 \else : #1 \fi #2,#1}
%凡例用BOX
\newtcolorbox[auto counter,number within=part]{papabox}[2][]{%
attach boxed title to top left, boxed title style={size=small,colback=black},breakable,
colback=black!5!white,colframe=black!75!black,fonttitle=\bfseries\sffamily,
title=凡例: #2 #1}
%ここからテスト
\newtcolorbox{pabox}[2][]{enhanced, attach boxed title to top left={xshift=0cm,yshift=-2mm}, 
breakable,fonttitle=\bfseries,varwidth boxed title=0.7\linewidth, 
colback=black!5!white,colframe=black,
boxed title style={boxrule=0.75mm,colframe=white, borderline={0.1mm}{0mm}}, 
borderline={0.1mm}{0.75mm} ,
title=凡例:{#2},#1}



\usepackage{tikz,lipsum} 
\tcbuselibrary{skins,breakable}
\colorlet{colexam}{red!5!black}
% Preamble: 
\usepackage{tikz,lipsum} 
\tcbuselibrary{skins,breakable} 
\newcounter{example} 
%Point用colorbox
\newtcolorbox[use counter=example]{hipoint}[2][]{%
empty,title={#2 },attach boxed title to top left, boxed title style={empty,size=minimal,toprule=2pt,top=4pt, overlay={\draw[colexam,line width=2pt] ([yshift=-1pt]frame.north west)--([yshift=-1pt]frame.north east);}}, coltitle=colexam,fonttitle=\bfseries, before=\par\medskip\noindent,parbox=false,boxsep=0pt,left=0pt,right=3mm,top=4pt, breakable,pad at break*=0mm,vfill before first, overlay unbroken={\draw[colexam,line width=1pt] ([yshift=-1pt]title.north east)--([xshift=-0.5pt,yshift=-1pt]title.north-|frame.east) --([xshift=-0.5pt]frame.south east)--(frame.south west); }, overlay first={\draw[colexam,line width=1pt] ([yshift=-1pt]title.north east)--([xshift=-0.5pt,yshift=-1pt]title.north-|frame.east) --([xshift=-0.5pt]frame.south east); }, overlay middle={\draw[colexam,line width=1pt] ([xshift=-0.5pt]frame.north east) --([xshift=-0.5pt]frame.south east); }, overlay last={\draw[colexam,line width=1pt] ([xshift=-0.5pt]frame.north east) --([xshift=-0.5pt]frame.south east)--(frame.south west);},%
}
\newtcolorbox[]{ansbox}[2][]{%
empty, coltitle=black!75!black,fonttitle=\bfseries, 
borderline north={0.5mm}{0pt}{black!80!white}, title=#2,
bottomrule=0mm, 
titlerule style={black}  }

\begin{document}

\section{クラスの応用}
\begin{pabox}{class-X}

ここで学生個人のデータを管理することを考えましょう。

学生は学生番号と履修科目と得点を管理しています。履修科目には必須科目(数学、英語、国語)とコース別のAcourse(絵画、音楽)、Bcourse(会計、情報)の選択必須科目があり、どちらかを選択しなければならないことを考えます。
また、学生の管理するデータを入力することと、学生ごとの合計点を計算する機能が必要だとします。

まずは必須科目を管理する機能と必須科目だけを5人分入力し、それぞれの合計点を出力するプログラムを作成します。
これをクラスとしてまとめるため、共通科目だけを扱う機能を次のように記載します。
\end{pabox}
\begin{codebox2}{一つのクラス}
\begin{listing}{1}
class Student: #共通科目を取り扱うためのクラス
    std_subjects = ["Mathematics" ,"English","Japanese"]
    def input_number(self):
            self.number = input( "number = " )
    def input_scores(self):
            self.scores = []
            input_data(self.std_subjects,self.scores)
        #データを入力するための手続きを関数としてまとめておきました。
        #エラー処理およびテキスト入力された得点を数値に変換する手続き

    def sum_of_scores(self):
            self.summary = 0
            for score in self.scores:
                self.summary = self.summary + score
        return self.summary

    # クラスの宣言はここまで
#ここでインデントを戻す
def input_data(subjects,scores):
    for s in subjects:
        x = input( s + " = " )
        y = int ( x )
        if y > 100 or y < 0 :
            y = -1
            print ( "エラー" )
        scores.append( y )

#グローバル変数(リスト)中身はStudentクラスの実体を格納する
student_list = []

#Studentクラスの実体を5名分追加
for n in range(5):
	student_list.append( Student() )

#学生の番号と得点の入力と合計点の出力
for std in student_list:
    std.input_number()
    std.input_scores()
    print(std.sum_of_scores())
\end{listing}
\end{codebox2}

\begin{pabox}{class-2}

凡例class-2

凡例class-1の共通機能を基にして、Acourse、Bcourseの選択必須科目の機能をを追加します。

Acourseの学生3名とBcourseの学生2名の得点を入力して(各コースの順番はばらばらでも可能)それぞれの学生の合計点とAcourseを選択している学生の平均点、Bcourseを選択している学生の平均点を計算するプログラムです。
\end{pabox}
\begin{codebox2}{クラスの継承}
\begin{listing}{1}

class Student: #共通科目を取り扱うためのクラス
    std_subjects = ["Mathematics" ,"English","Japanese"]
    def input_number(self):
            self.number = input( "number = " )
    def input_scores(self):
            self.scores = []
            input_data(self.std_subjects,self.scores)
        #データを入力するための手続きを関数としてまとめておきました。
        #エラー処理およびテキスト入力された得点を数値に変換する手続き

    def sum_of_scores(self):
            self.summary = 0
            for score in self.scores:
                self.summary = self.summary + score
        return self.summary

def input_data(subjects,scores):
    for s in subjects:
        x = input( s + " = " )
        y = int ( x )
        if y > 100 or y < 0 :
            y = -1
            print ( "エラー" )
        scores.append( y )
#ここまでは凡例「class-1」の24行目までと同一です。

class AcourseStudent(Student):
    acourse_subjects = ["picture","music"]
    def input_scores(self):
        super().input_scores()
        input_data(self.acourse_subjects,self.scores)
class BcourseStudent(Student):
    bcourse_subjects = ["accounting","information"]
    def input_scores(self):
        super().input_scores()
        input_data(self.bcourse_subjects,self.scores)

student_list = []
#凡例class-1の学生の追加部分を次のように書き換え
student_list.append( AcourseStudent() )
student_list.append( AcourseStudent() )
student_list.append( BcourseStudent() )
student_list.append( BcourseStudent() )
student_list.append( AcourseStudent() )

#学生の番号と得点の入力と合計点の出力
for std in student_list:
    std.input_number()
    std.input_scores()

for std in student_list:
    print("number = " , std.number , " summary = ", \
       std.sum_of_scores())
\end{listing}
\end{codebox2}

おまけpythonによるguiプログラミング(冬休みの宿題)

\newpage

\begin{verbatim}
pip install pyqt5
\end{verbatim}

\begin{pabox}{class-2}
\begin{listing}{1}
# pylint: disable=missing-docstring
# pylint: disable=no-name-in-module
# pylint: disable=unused-import
import sys
from PyQt5.QtWidgets import (QWidget, QPushButton, QApplication)

class Example(QWidget):
    def __init__(self):
        super().__init__()
        self.btn = QPushButton('こんにちは', self) #ボタンの生成
        self.btn.clicked.connect(self.push)  
        self.btn.setGeometry(0,0,300,200)
        self.setGeometry(300, 300, 300, 200)
        self.show()

    def push(self):
        self.btn.setText("押されました")

app = QApplication(sys.argv)
ex = Example()
sys.exit(app.exec_())
\end{listing}
\end{pabox}

\end{document}